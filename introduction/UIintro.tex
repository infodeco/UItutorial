% UIintro.tex
% (c) 2018 by the authors
\documentclass{article}

\usepackage{amsmath}

\title{An introduction to information decompositions}

\author{Johannes Rauh, \dots}

\begin{document}

\maketitle{}

\tableofcontents{}

\section{Information theory}
\label{sec:infotheory}

\section{Bivariate information decompositions}
\label{sec:infodeco}

When Shannon proposed to use entropy in order to quantify information, he had in mind a very restricted setting of
communication over a noisy channel~\cite{Shannon48:A_Mathematical_Theory_of_Communication}.  Since then, the use of entropic quantities as measures of information has been
greatly expanded, and they are nowadys used successfully in a large variety of applications, such as statistical
physics, the study of complex systems, analysis of neural networks, and machine learning \cite{}.
In particular, transfer entropy is used as a tool to study causality in dynamic systems \cite{}, and
mutual information is used successfully as a selection criterion in feature selection \cite{}.

In many applications, the effort of estimating and approximating entropic quantities (which is far from negligeable) is
out-weighed by the performance benefit.

Despite this huge success of information theory, there are still many open questions about the nature of information.
In particular, since information is not a conservation quantity, it is difficult to describe how information is
distributed over composite systems.  Clearly, different subsystems may have exclusive information, or they may have
redundant information.  Third, synergy effects complicate the analysis: It may happen that some information is not known
to any subsystem, but can only be reconstructed from the knowledge of the complete system.  An example is the
digit sum of an integer, which can only be computed when all digits are known.  Such synergy effects are used
heavily in cryptography, where the goal is that the encrypted message alone contains no information about the original
message without knowing the key.

So far, there is no consensus on how to define information quantities that describe the unique, shared and synergistic
proportion of joint information, even though there have been several proposals: For example, the
co-information~\cite{Bell2003:Coinformation}
\begin{equation*}
  CoI(S;X;Y) := I(S; X) - I(S; X|Y),
\end{equation*}
which equals the negative interaction information~\cite{McGill54:interaction-information}, has been proposed as a
measure of shared and synergistic information: If $CoI(S;X;Y)>0$, then $X$ and $Y$ carry redundant information about~$S$
(knowing $Y$ decreases the mutual information between $S$ and~$X$), and if $CoI(S;X;Y)>0$, then $X$ and $Y$ carry
synergistic information about~$S$ (the knowledge of $Y$ increases the mutual information between $S$ and~$X$).  However,
it is generally agreed that redundancy and synergy effects can coexist, and in this case the co-information can only
measure which of these two effects is larger.

As another example, the conditional mutual information $I(S; X|Y)$ is often used to measure the influence of $X$ on~$S$,
discounting for the influence of $Y$ on~$S$.  For instance, in the context of a stochastic dynamical system of three
random variables $(X_{t},Y_{t},S_{t})_{t=0}^{\infty}$, the transfer entropy takes the form of a conditional mutual
information:
\begin{equation*}
  I(S_{t}; X_{t}|Y_{0},\dots,Y_{t}).
\end{equation*}
The goal of the transfer entropy is to measure the influence of the value of $X$ at time~$t$ on the value of $S$ at
time~$t$, discounting for the influence of the past of $Y$ on~$S$.  This influence is sometimes even interpreted as a
causal effect.  However, it has been noted that the transfer entropy also includes synergistic effects, and this is true
more generally for the conditional mutual information $I(S; X|Y)$: If $X$ and~$Y$ are both independent of $S$ but
together interact synergistically to produce~$S$, then the conditional mutual information will be large.  In such a
case, the sum % of the two transfer entropies
\begin{equation*}
%  T(X\to S|Y)_{t} + T(Y\to S|X)_{t}
  I(S; X|Y) + I(S; Y|X)
\end{equation*}
will double-count this synergistic effect and may be larger than $I(S; X,Y)$ (and thus overestimate the total effect that
$X$ and $Y$ have on~$S$).  This second example is related to the first example, as
% Under the assumption that $S,X,Y$ are independent of $Y_{0},\dots,Y_{t}$ and independent of $X_{0},\dots,X_{t}$, then
\begin{equation*}
%  T(X\to S|Y)_{t} + T(Y\to S|X)_{t} =
  I(S; X|Y) + I(S; Y|X)
  = I(S; X,Y) - CoI(S;X;Y).
\end{equation*}

A lot of research has focussed on finding an information measure for a single aspect, most notably a measure for
synergy~\cite{}.  The seminal paper~\cite{WilliamsBeer:Nonneg_Decomposition_of_Multiinformation} proposed a more
principled approach, namely to find a complete decomposition of the total mutual information $I(S; X_{1},\dots,X_{k})$
about a signal~$S$ that is distributed among a family of random variables~$X_{1},\dots,X_{k}$ into a sum of non-negative
terms with a well-defined interpretation corresponding to the different ways in which information can have aspects of
redundancy, unique or synergistic information.  In the case~$k=2$ (called the \emph{bivariate case}), the decomposition is of the form
\begin{subequations}
  \begin{multline}
  I(S;X_1X_2) = \underbrace {{SI}(S;X_1,X_2)}_{\text{shared}} + \underbrace {CI(S; X_1,X_2)}_{\text{complementary}} \\
  + \underbrace {UI(S;X_1\backslash X_2)}_{\text{unique ($X_1$ wrt $X_2$)}} + \underbrace {UI(S;X_2\backslash X_1)}_{\text{unique ($X_2$ wrt $X_1$)}}, \label{eq:bivariate1}
\end{multline}
where $SI(S;X_{1},X_{2})$, $UI(S;X_{1}\backslash X_{2})$, $UI(S;X_{2}\backslash X_{1})$, and $CI(S;X_{1},X_{2})$ are nonnegative\footnote{Information decompositions with negative terms have also been proposed, although their interpretation is much more difficult \cite{}.} functions that depend continuously\footnote{Non-continous information decompositions have also been proposed, see \cite{}.} on the joint distribution of $(S,X_{1},X_{2})$.
Furthermore, these functions are required to satisfy
\begin{align}
    I(S;X_1) &= SI(S;X_1,X_2) + UI(S;X_1\backslash X_2),  \label{eq:bivariate2}\\
    I(S;X_2) &= SI(S;X_1,X_2) + UI(S;X_2\backslash X_1).  \label{eq:bivariate3}
\end{align}
\end{subequations}

Combining these equations, it follows, that the co-information can be written as the difference of redundant and synergistic information:
\begin{equation*}
  CoI(S;X_1,X_2) = I(S;X_1) - I(S;X_1|X_2)
  = SI(S;X_1,X_2) - CI(S; X_1,X_2). 
\end{equation*}
which agrees with the intuition developed above.  Moreover, the conditional mutual information satisfies
\begin{equation*}
  I(S;X_1|X_2) = I(S;X_{1}X_{2}) - I(S;X_{2})
  = CI(S;X_1,X_2) + UI(S;X_1\backslash X_2),
\end{equation*}
which, again, agrees with what has been said above.

While the framework of~\cite{WilliamsBeer:Nonneg_Decomposition_of_Multiinformation} is intriguing, so far, there is no
consensus on the definition of the corresponding functions for shared, unique and synergistic information.  Here, we
follow the approach from~\cite{BROJA13:Quantifying_unique_information} and use the functions $SI$, $UI$ and~$CI$ defined
there.  Note that this approach only covers the case~$k=2$.  We will see that even this case proves useful\dots

At this point one can ask whether information decompositions can only provide theoretical insight, or whether sensible
information measures $SI$, $UI$ and~$CI$ are useful in applications.  To argue for the second answer, let us focus on
the example of feature selection.  Suppose that the task is to predict a classification variable $C$, using a subset of
some larger set of features $\{F_{i}\}$.  Suppose that features $F_{1},\dots,F_{k}$ have already been selected and that
we are looking for the next best feature $F_{k}$ to add to our list.  A common information-theoretic criterion suggests
to add the feature $F$ that maximizes the mutual information~$I(C; F_{1},\dots,F_{k},F)$.  Equivalently, using the chain
rule, one can maximize the conditional mutual information $I(C; F|F_{1},\dots,F_{k})$.  As explained above, this
conditional mutual information measures the information that $F$ has about~$C$ in the presence of~$F_{1},\dots,F_{k}$,
and it contains unique aspects as well as synergistic aspects:
\begin{equation*}
  I(C; F|F_{1}\dots F_{k}) = UI(C; F\setminus F_{1}\dots F_{k}) + CI(C;F, (F_{1}\dots F_{k}))
\end{equation*}
One could say that this is good, since both the unique contribution as well as the synergistig contribution are wellcome
contributions on the way to increase the total mutual information about~$C$.  However, we argue that there are instances
where it is better to optimize just the unique contribution.  The motivation comes from secondary objectives in feature
selection, beyond striving to minimize the prediction error: Namely, we claim that a system that largely relies on
synergistic effects tends to be much more complicated to understand and much less robust.

The first claim is easy to explain: A common definition of complexity states that a composite system is complex if it is
``more than the sum of its parts,'' that is, if its parts interact synergistically together.

The second claim arguably depends on the setting, as is often true in robustness analysis, where any measure of
robustness depends on the possible perturbations that one has in mind.  We believe that the claim is true in the setting
where features are perturbed independently of each other.  For example, when the features are extracted from noisy data,
the value of each feature is found correctly only with a certain probability.  If the features contribute separately to
the classification, then each correctly identified feature improves the prediction.  However, when several features
interact synergistically, then an error in the identification of one of these features makes the values of all features
unhelpful.

\section{Unique information from decision problems}
\label{sec:UIBROJA}

\section{Alternative information decompositions}
\label{sec:alt_infodeco}

\section{Properties of information decompositions}
\label{sec:properties}

\bibliographystyle{IEEEtranS}
\bibliography{infodeco}

\end{document}

%%% Local Variables: 
%%% mode: latex
%%% TeX-master: t
%%% End: 
